
\section{Introdução}\label{sec:introducao}

\fontsize{12}{12}\selectfont
    \par Na implementação do código image8bit.c, foram criadas diversas funções para manipular imagens
        de diferentes formas. Para além de implementar, é necessário analisá-las para garantir um 
        código mais eficiente. Para esse propósito, foram analisados os algoritmos das funções
        ImageLocateSubImage() e ImageBlur().

    \par O ficheiro image8bit.c contêm múltiplas funções para manipular imagens PGM(Portable Gray Map):

    \begin{itemize}
        \item \textbf{ImageValidPos} - Analisa se uma posição que corresponde a um píxel da imagem referida
            encontra-se dentro da mesma
        \item \textbf{ImageValidRect} - Analisa se as dimensões correspondentes a um retângulo se encontram
            dentro da imagem ou não
        \item \textbf{ImageGetPixel e ImageSetPixel} - Responsáveis por obter o valor do pixel
            na respetiva posição e alterar esse valor por um outro.
        \item \textbf{ImageNegative} - Inverte os valores de cada pixel $(maxval - pixelValue)$
        \item \textbf{ImageThreshold} - Converte os valores de cada píxel para 0 ou para maxval
            se o mesmo for inferior ou superior ou igual respetivamente ao valor definido como argumento
        \item \textbf{ImageBrighten} - Converte os valores de cada píxel multiplicando por um fator
        \item \textbf{ImageRotate} - Cria uma imagem nova com os valores dos pixeis da original
            trocada de modo a obter uma imagem rodada 90º no sentido contrário dos ponteiros do relógio
        \item \textbf{ImageMirror} - Devolve uma imagem espelhada horizontalmente
        \item \textbf{ImageCrop} - Recorta uma parte da imagem e devolve-a noutra imagem
        \item \textbf{ImagePaste} - Insere uma imagem por cima de outra, certificando que
            a imagem a inserir seja inferior à imagem onde será inserida
        \item \textbf{ImageBlend} - Junta 2 imagens, confirmando que a que se vai juntar seja 
         inferior à outra e é aplicado um valor $alpha$ entre 0.0 a 1.0
        \item \textbf{ImageMatchSubImage} - Compara se a partir de uma dada posição (x,y) 
            da imagem1 a imagem2 é igual
        \item \textbf{ImagemLocateSubImage} - Chama a função ImageMatchSubImage em cada píxel da
            imagem1 para comparar e define 2 ponteiros(px e py) para esse mesmo píxel
        \item \textbf{ImageBlur} - Aplica um filtro na imagem para criar uma desfocada
    \end{itemize}

    